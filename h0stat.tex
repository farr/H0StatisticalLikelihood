\documentclass[modern]{aastex62}
\usepackage[utf8]{inputenc}

\usepackage{acronym}
\usepackage{amsmath}
\usepackage{amssymb}

\acrodef{EM}{electromagnetic}
\acrodef{GW}{gravitational wave}

\newcommand{\dd}{\mathrm{d}}
\newcommand{\dEM}{d_{\mathrm{EM}}}
\newcommand{\diff}[2]{\frac{\dd #1}{\dd #2}}
\newcommand{\dGW}{d_{\mathrm{GW}}}
\newcommand{\Ngal}{N_{\mathrm{gal}}}
\newcommand{\OGW}{\Omega_{\mathrm{GW}}}
\newcommand{\Pdet}{P_{\mathrm{det}}}
\newcommand{\thetaEM}{\theta_{\mathrm{EM}}}
\newcommand{\thetaGW}{\theta_{\mathrm{GW}}}

\begin{document}

\title{A Derivation of the Likelihood Function for a Statistical $H_0$
Measurement}

\author{Will M. Farr}

\affiliation{Department of Physics and Astronomy, Stony Brook University, Stony
Brook NY 11794, United States}

\affiliation{Center for Computational Astrophysics, Flatiron Institute, New York
NY 10010, United States}

\email{will.farr@stonybrook.edu}

\email{wfarr-vscholar@flatironinstitute.org}

\begin{abstract}
%
  I derive the likelihood function for a ``statistical'' measurement of the
  Hubble constant that combines a distance measured by a \ac{GW} observation and
  a catalog of redshifts from possible host galaxies measured by \ac{EM}
  observations.
%
\end{abstract}

\section{Likelihood}

We have two data sets: a \ac{GW} set $\dGW$, corresponding to a single detected
\ac{GW} event, and an \ac{EM} data set $\dEM$, corresponding to a survey of
galaxies that overlaps on the sky with the \ac{GW} event's inferred sky
position.  A likelihood function describes the generation of the \ac{GW} data
given \ac{GW} parameters $\thetaGW$ such as masses, spins, etc:
%
\begin{equation}
    \mathcal{L}\left( \thetaGW \right) \equiv p\left( \dGW \mid \thetaGW \right).
\end{equation}
%
Crucially, the luminosity distance to the event, $d_L$, and the position of the
event on the sky $\OGW$ are among the \ac{GW} parameters.  Similarly, we have an
\ac{EM} likelihood that describes the generation of the survey data given
parameters for the galaxies in the survey:
%
\begin{equation}
    \mathcal{L} \left( \thetaEM \right) \equiv p\left( \dEM \mid \thetaEM \right).
\end{equation}
%
Among the parameters for the \ac{EM} survey are the redshifts of the galaxies
$\left\{ z_i \mid i = 1, \ldots, \Ngal \right\}$ and also the positions of the
galaxies on the sky $\left\{ \Omega_i \mid i = 1, \ldots, \Ngal \right\}$.

The joint likelihood for the two sets of observations is just the product of the
individual likelihoods because the data generating processes in each instrument
are independent of each other:
%
\begin{equation}
    p\left( \dGW, \dEM \mid \thetaGW, \thetaEM \right) =  p\left( \dGW \mid \thetaGW \right) p\left( \dEM \mid \thetaEM \right).
\end{equation}
%
To measure the Hubble constant, we will not be concerned with the masses, spins,
etc, of the \ac{GW} source, nor with \ac{EM} data besides the redshifts and sky
positions, so let us impose some standard priors and integrate these variables
out.  This leaves us with a \emph{marginal} likelihood for each observation
expressed in terms of the luminosity distance, sky positions, and redshifts:
%
\begin{multline}
    p\left( \dGW \mid d_L, \OGW \right) p\left( \dEM \mid \left\{ z_i, \Omega_i \mid i = 1, \ldots, \Ngal \right\} \right) = \\ \int \dd \bar{\theta}_{\mathrm{GW}} \, \dd \bar{\theta}_{\mathrm{EM}} \, p\left( \bar{\theta}_{\mathrm{GW}} \right) p\left( \bar{\theta}_{\mathrm{EM}} \right) p\left( \dGW \mid \thetaGW \right) p\left( \dEM \mid \thetaEM \right),
\end{multline}
%
where the ``bar'' indicates the set of ignorable parameters and $p\left(
\bar{\theta}_{\mathrm{GW}} \right)$ and $p\left( \bar{\theta}_{\mathrm{EM}}
\right)$ are the priors we impose on these parameters.

Now let us impose the constraint that one of the galaxies in the survey must
have hosted the \ac{GW} event; suppose that we weight the choice of hosts by a
set of weights $\left\{ w_i \mid i = 1, \ldots, \Ngal \right\}$.  This
constraint will tie together both pairs $\OGW$ and $\Omega_i$ and $d_L$ and
$z_i$; the relation between $d_L$ and $z_i$ depends on $H_0$ (and other
cosmological parameters).  We can use these relationships to marginalize over
(i.e.\ eliminate) $d_L$ and $\OGW$ from the likelihood, obtaining for a single
galaxy host, $i$
%
\begin{multline}
    p\left( \dGW, \dEM \mid i, z_i, \Omega_i, H_0 \right) \equiv \\ \int \dd d_L \, \dd \OGW \, p\left( \dGW \mid d_L, \OGW \right) p\left( \dEM \mid \left\{ z_i, \Omega_i \mid i = 1, \ldots, \Ngal \right\} \right)  \\ \times \delta\left( d_L - d_L\left( z_i, H_0 \right) \right) \delta \left( \OGW - \Omega_i \right).
\end{multline}
%
The integrals are trivial, leaving
%
\begin{multline}
    p\left( \dGW, \dEM \mid i, z_i, \Omega_i, H_0 \right) = \\ p\left( \dGW \mid d_L\left( z_i, H_0 \right), \Omega_i \right) p\left( \dEM \mid \left\{ z_i, \Omega_i \mid i = 1, \ldots, \Ngal \right\} \right).
\end{multline}

In the absence of other information, we do not know which galaxy hosted the
\ac{GW} event, so we should marginalize over the choice of galaxy, $i$, with the
correct weights:
%
\begin{multline}
    p\left( \dGW, \dEM \mid \left\{ z_i, \Omega_i \mid i = 1, \ldots, \Ngal \right\}, H_0 \right) = \\ \sum_{i = 1}^{\Ngal} w_i  p\left( \dGW \mid d_L\left( z_i, H_0 \right), \Omega_i \right) p\left( \dEM \mid \left\{ z_i, \Omega_i \mid i = 1, \ldots, \Ngal \right\} \right).
\end{multline}
%
The galaxy redshifts and sky positions are also nuisance parameters in an
inference of $H_0$ so we should marginalize over them with a suitable choice of
prior.  A natural prior would be to assume that galaxies are uniformly
distributed in comoving volume and on the sky, so
%
\begin{equation}
    p\left( z_i, \Omega_i \right) \dd z_i \, \dd \Omega_i = \frac{1}{4\pi V_\mathrm{max}} \diff{V}{z_i} \, \dd z_i \, \dd \Omega_i
\end{equation}
%
where $V_\mathrm{max}$ is some suitably large comoving volume that encompasses
all galaxies in the survey.  At sufficiently small redshifts, $p\left( z_i
\right) \propto z_i^2$.  Note that at \emph{all} redshifts, $p\left( z_i
\right)$ is independent of $H_0$.

Finally, we arrive at the marginal likelihood for $H_0$ given the joint
observations:
%
\begin{multline}
    \label{eq:H0-marginal-likelihood}
    p\left( \dGW, \dEM \mid H_0 \right) = \sum_{i=1}^{\Ngal} w_i \int \dd z_i \, \dd \Omega_i \, p\left( z_i, \Omega_i \right) p\left( \dGW \mid d_L\left( z_i, H_0 \right), \Omega_i \right) \\ \times p\left( \dEM \mid \left\{ z_i, \Omega_i \mid i = 1, \ldots, \Ngal \right\} \right)
\end{multline}

\subsection{Selection Effects}

So far, we have been ignoring selection effects.  Selection effects can be
modeled by including a factor, $\beta\left( H_0 \right)$ ensuring that the
marginal likelihood integrates to one over all \emph{detectable} data sets
\citep{Mandel2018}.  In other words, we write
%
\begin{multline}
    p\left( \dGW, \dEM \mid H_0 \right) = \frac{1}{\beta\left( H_0 \right)} \sum_{i=1}^{\Ngal} w_i \int \dd z_i \, \dd \Omega_i \, p\left( z_i, \Omega_i \right) p\left( \dGW \mid d_L\left( z_i, H_0 \right), \Omega_i \right) \\ \times p\left( \dEM \mid \left\{ z_i, \Omega_i \mid i = 1, \ldots, \Ngal \right\} \right)
\end{multline}
%
such that
\begin{equation}
    \int \dd \dGW \, \dd \dEM \, \Pdet\left( \dGW \right) \Pdet\left( \dEM \right) p\left( \dGW, \dEM \mid H_0 \right) = 1,
\end{equation}
%
where $\Pdet\left(d\right)$ refers to the probability that a particular data set
$d$ will result in a ``detection'' of a \ac{GW} (typically, $\Pdet$ is either 0
or 1, depending on the output of a pipeline applied to the data; it serves to
truncate the integral to detectable data sets only).  Thus,
%
\begin{multline}
    \label{eq:beta-H0-full-def}
    \beta\left( H_0 \right) =  \int \dd \dGW \, \dd \dEM \, \Pdet\left( \dGW \right) \Pdet\left( \dEM \right) \\ \times \sum_{i=1}^{\Ngal} w_i \int \dd z_i \, \dd \Omega_i \, p\left( z_i, \Omega_i \right) p\left( \dGW \mid d_L\left( z_i, H_0 \right), \Omega_i \right) \\ \times p\left( \dEM \mid \left\{ z_i, \Omega_i \mid i = 1, \ldots, \Ngal \right\} \right).
\end{multline}
%
Note that the issue of selection effects has nothing to do with how
``detectable'' the observed events are (by definition, in the absence of a
stochastic detection pipeline, $\Pdet\left( d_{\mathrm{GW},\mathrm{observed}}
\right) \equiv 1$), but rather how the detectability of the \emph{population} of
\ac{GW} events varies with $H_0$.

Current catalogs are complete (i.e.\ $\Pdet\left( \dEM \right) = 1$) for any
\ac{GW} detection assuming reasonable values of $H_0$, so here we will focus on
$\Pdet\left( \dGW \right)$.  Performing the integral over $\dEM$ (remember that
the \ac{EM} likelihood is normalised), we obtain
%
\begin{multline}
    \beta\left( H_0 \right) =  \int \dd \dGW \, \Pdet\left( \dGW \right) \\ \times \sum_{i=1}^{\Ngal} w_i \int \dd z_i \, \dd \Omega_i \, p\left( z_i, \Omega_i \right) p\left( \dGW \mid d_L\left( z_i, H_0 \right), \Omega_i \right).
\end{multline}
%
The integrals in each term of the sum are now all identical (only the \ac{EM}
data distinguishes the different galaxies) and the sum over weights is
normalized, so we are left with
%
\begin{equation}
     \beta\left( H_0 \right) =  \int \dd \dGW \, \dd z \, \dd \Omega \, \Pdet\left( \dGW \right)  p\left( z, \Omega \right) p\left( \dGW \mid d_L\left( z, H_0 \right), \Omega \right).
\end{equation}
%
We can change variables from $z$ to $d_L$ in the integral, obtaining
%
\begin{equation}
    \beta\left( H_0 \right) =  \int \dd \dGW \, \dd d_L \, \dd \Omega \, \Pdet\left( \dGW \right)  p\left( d_L, \Omega \right) p\left( \dGW \mid d_L, \Omega \right).
\end{equation}
%
(Recall that $p\left( z, \Omega \right) \, \dd z \, \dd \Omega = p\left( d_L,
\Omega \right) \, \dd d_L \, \dd \Omega$.)  But this latter integral is
independent of $H_0$ so long as the mass distribution assumed in the \ac{GW}
prior is scale-free or the relevant redshift range is such that the detection
efficiency is independent of the redshifting of mass in the population.  So,
$\beta\left( H_0 \right) = \mathrm{const}$ as long as \ac{EM} catalogs are
complete for all observed \ac{GW} events.  Under these assumptions, the
detection efficiency of LIGO is not dependent on the Hubble constant.

Eventually, \ac{EM} catalogs will not be complete for all \ac{GW} events.  If
only the set of \ac{GW} events for which the \ac{EM} catalog is complete is used
to estimate $H_0$ then $\beta\left( H_0 \right)$ can be computed by returning to
Eq.\ \eqref{eq:beta-H0-full-def}.  Alternately, the full set of \ac{GW} events
can be used, and a fictitious catalog members can be added with weights
corresponding to the incompleteness fraction at various redshifts and in
different directions; the limit of this procedure yields the formulas in
\citet{Fishbach2018} for incomplete catalogs.

\section{Practical Considerations}

Evaluating Eq.\ \eqref{eq:H0-marginal-likelihood} requires the ability to
compute the \ac{GW} marginal likelihood at arbitrary $d_L$ and $\Omega$.
Happily, a standard data product provided by LIGO \citep{Singer2016} can be
adapted to produce a function that approximates the marginal \ac{GW} likelihood
at fixed locations on the sky corresponding to HEALPix pixels
\citep{Gorski2005}.  The method in \citet{Singer2016} fits the \emph{posterior}
over distance and sky position from a LIGO analysis using a per-pixel ansatz
where
%
\begin{equation}
    \label{eq:gtd-ansatz}
    p\left( d_L, \Omega \mid \dGW \right) = \frac{\rho_i}{A} \frac{N_i}{\sqrt{2\pi} \sigma_i} \exp\left[ -\frac{1}{2} \left(\frac{d_L-\mu_i}{\sigma_i}\right)^2 \right] d_L^2,
\end{equation}
%
with $\Omega$ in HEALPix pixel $i$; $N_i$, $\rho_i$, $\mu_i$, and $\sigma_i$
per-pixel fitting parameters; and $A$ the area of the pixel.  Since the standard
LIGO distance prior is proportional to $d_L^2$, this means that the \ac{GW}
marginal likelihood when $\Omega$ is in pixel $i$ is (up to ignorable constants)
%
\begin{equation}
    p\left( \dGW \mid d_L, \Omega \right) \propto \frac{\rho_i}{A} \frac{N_i}{\sqrt{2\pi} \sigma_i} \exp\left[ -\frac{1}{2} \left(\frac{d_L-\mu_i}{\sigma_i}\right)^2 \right].
\end{equation}
%
(Note the missing factor of $d_L^2$ relative to Eq.\ \eqref{eq:gtd-ansatz}.)

If we are in the linear regime, where
%
\begin{equation}
    d_L = \frac{c z}{H_0}
\end{equation}
%
to sufficient accuracy, the \ac{EM} likelihood is Gaussian in the redshift,
%
\begin{equation}
    p\left( \dEM \mid z_i, \Omega_i \right) \propto \frac{1}{\sqrt{2\pi} \sigma_{z,i}} \exp\left[ -\frac{1}{2} \left( \frac{z - z_{i,\mathrm{obs}}}{\sigma_{i,z}} \right)^2 \right],
\end{equation}
%
we assume that the \ac{EM} data determines the sky position of each galaxy
precisely, then the marginalization integrals over redshift and sky position in
Eq.\ \eqref{eq:H0-marginal-likelihood} can be performed analytically, as we now
show. The result, when galaxy $i$ lies inside pixel $k_i$, is
%
\begin{multline}
     p\left( \dGW, \dEM \mid H_0 \right) = \\ \frac{3 H_0^3}{z_\mathrm{max}^3} \sum_{i=1}^{\Ngal} \frac{w_i \rho_{k_i} N_{k_i}}{A \sqrt{2\pi} \Sigma} \exp\left[ -\frac{\left(H_0 \mu_{k_i} - c z_{i,\mathrm{obs}}\right)^2}{2 \Sigma^2} \right] \\ \times \left( \frac{\sigma_{k_i}^2 \sigma_{i,z}^2}{\Sigma^2} + \frac{\left( c \mu_{k_i} \sigma_{i,z}^2 + H_0 \sigma_{k_i}^2 z_{i,\mathrm{obs}} \right)^2}{\Sigma^4} \right),
\end{multline}
%
with
%
\begin{equation}
    \Sigma^2 = H_0^2 \sigma_{k_i}^2 + c^2 \sigma_{i,z}^2.
\end{equation}
%
The galaxy catalog is actually volume-limited (we are assuming that it is
complete over the entire \ac{GW} sensitive volume), so the factor
$H_0/z_\mathrm{max}$ is actually independent of $H_0$ and equal to
$c/d_\mathrm{max}$.

It is reasonable to truncate the sum above by ignoring galaxies for which
$\rho_{k_i}$ (the integrated per-pixel \ac{GW} posterior probability) is
sufficiently small.

\subsection{The Uninformative Limits}

There are two different limits in which the \ac{EM} measurement is
uninformative.  The first limit is when the individual redshift measurements are
too uncertain to meaningfully constrain $H_0$.  In this limit
%
\begin{equation}
  H_0^2 \sigma_{k_i}^2 \ll c^2 \sigma_{z,i}^2 \to \infty
\end{equation}
%
in the relevant pixels.  Since the $c^2 \sigma_{i,z}^2$ term dominates $\Sigma$
and also eliminates the exponential, the likelihood function for $H_0$ becomes
constant and the measurement is uninformative about $H_0$.

The second limit is when the number of galaxies in the \ac{GW} localization
region is large.  TODO: finish this terrible algebra.

\bibliography{h0stat}

\end{document}
